\documentclass{PionpillNote-book}
\usetikzlibrary {intersections,through,arrows.meta,graphs,shapes.misc,positioning,shapes.misc,positioning,calc}
\usetikzlibrary{animations}
\usetikzlibrary {shapes.geometric}
\usetikzlibrary {animations}
\usetikzlibrary {shapes.multipart}
\usetikzlibrary {positioning}
\usetikzlibrary {fit,shapes.geometric}
\usetikzlibrary {automata}
\usetikzlibrary {quotes}
\usetikzlibrary {matrix}
\usetikzlibrary {backgrounds}
\usetikzlibrary {scopes}
\usetikzlibrary {calc}
\usetikzlibrary {intersections}
\usetikzlibrary {svg.path}
\usetikzlibrary {decorations}
\usetikzlibrary {patterns}
\usetikzlibrary {decorations.pathmorphing}
\usetikzlibrary {shadows}
\usetikzlibrary {bending}

\title{算法笔记}
\author{
    Pionpill \footnote{笔名:北岸,电子邮件:673486387@qq.com,Github:\url{https://github.com/Pionpill}} \\
    本文档为算法学习笔记\footnote{开源地址: \url{https://github.com/Pionpill/Notobook-Algorithms}},主要参考书籍为 <<算法(第4版)>>\footnote{<<Algorithms forth edition>> [美] Robert Sedgewick 2012 版}。
}

\date{\today}

\begin{document}

\pagestyle{plain}
\maketitle

\noindent\textbf{前言}

本文以<<算法4>>为参考,记录了绝大多数基础算法(查找,排序,树,动态规划...)思路与主要实现代码。采用 Java 语言实现,读者必须具备 Java 编程基础与算法理论基础。不同于原书,本文没有使用 algs4 包,完全采用 Java 语言及标准库中的工具实现。

本文书写环境:
\begin{itemize}
    \item OS: Window11
    \item IDE: VSCode(Latex), IDEA(Java)
    \item Java: JDK17
\end{itemize}

本文不会给出完整的算法实现,只对核心算法做解释,如果需要请前往项目地址查看具体实现方式。

关于跳过的第一章内容简述: 第一章主要以 Java 语言为基础,讲解了必要的语法与编程思想,如有必要,请看原文。

关于几个算法评价指标的说明:

\begin{itemize}
    \item 时间复杂度: 即运行的时间,有时会使用不标准的 O(n/2) 形式进行比较。
    \item 空间复杂度: 即占用的内存空间,有时会使用不标准的 O(n/2) 形式进行比较。
    \item 对于某些特定算法会引入其他指标: 如排序算法中的稳定性。
\end{itemize}

本文会采用 Java 语言的编写习惯,例如面向接口编程,同时在部分章节会分析 Java 标准库中一些算法的具体实现。

此外非常推荐一个项目: algorithm-visualizer
\footnote{\url{https://github.com/algorithm-visualizer/algorithm-visualizer}},可以动态演示算法过程。

此外,本书不能替代原文,仅局限于算法的实现,关于更多的重要的思想请查看原文。

\newpage

\tableofcontents

\newpage

\setcounter{page}{1} 
\pagestyle{fancy}

\chapter{排序}
\import{Contents/Sort}{Abstract.tex}
\import{Contents/Sort}{PrimarySort.tex}
\import{Contents/Sort}{MergeSort.tex}
\import{Contents/Sort}{QuickSort.tex}
\import{Contents/Sort}{HeapSort.tex}
\import{Contents/Sort}{Conclusion.tex}

\end{document}

