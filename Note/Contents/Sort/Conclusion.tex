\section{排序算法总结}

\subsection{通用排序算法比较}

前面说明的七种算法都属于通用排序算法,在 Java 中只要实现了 \texttt{Comparable} 接口就可以使用。这几种算法总结如下:

\begin{table}[H]
    \small
    \centering
    \caption{通用排序算法比较}
    \label{table:通用排序算法比较}
    \setlength{\tabcolsep}{2mm}
    \begin{tabular}{c|ccccp{2.25cm}p{2.25cm}p{2.25cm}}
        \toprule
        \textbf{排序算法} & \textbf{时间} & \textbf{时间(最坏)} & \textbf{空间} & \textbf{稳定性} & \textbf{优点} & \textbf{缺点} & \textbf{适用范围} \\
        \midrule
        冒泡排序 & $O(n^2)$ & $O(n^2)$ & $O(1)$ & 稳定 & - & - & - \\
        选择排序 & $O(n^2)$ & $O(n^2)$ & $O(1)$ & 不稳定 & - & - & - \\
        插入排序 & $O(n^2)$ & $O(n^2)$ & $O(1)$ & 稳定 & - & - & - \\
        \midrule
        希尔排序 & $O(n^{1.3})$ & $O(n^2)$ & $O(1)$ & 不稳定 & 比下有余 & 比上不足 & 备胎 \\
        \midrule
        归并排序 & $O(n \log_2 n)$ & $O(n \log_2 n)$ & $O(n)$ & 稳定 & 非常稳定 & 额外空间 & 数据量小 \\
        快速排序 & $O(n \log_2 n)$ & $O(n^2)$ & $O(\log_2 n)$ & 不稳定 & 缓存命中率高 & 时间不稳定 & 数据量大随机 \\
        堆排序 & $O(n \log_2 n)$ & $O(n \log_2 n)$ & $O(1)$ & 不稳定 & 对数级别运算 & 缓存命中率低 & 优先队列 \\
        \bottomrule
    \end{tabular}
\end{table}