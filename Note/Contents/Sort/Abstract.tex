排序算法引入了一个新的指标:稳定性,即保证排序前两个相等的数据其在序列中的先后位置顺序与排序后它们两个先后位置顺序相同。默认算法能保证稳定则稳定,除非你闲着没事干交换两个相同元素位置。

一些常用的数据符号:

\begin{table}[H]
    \centering
    \setlength{\tabcolsep}{4mm}
    \begin{tabular}{c|ccc}
        \toprule
        \textbf{符号} & \textbf{意义} \\
        \midrule
        n & 数组/元素集合长度 \\
        i,j & 元素下标 \\
        p & 指针 \\
        \bottomrule
    \end{tabular}
\end{table}

此外,本人默认采用如下的方式进行排序:
\begin{itemize}
    \item 数据结构: 本文默认使用数组存储元素。
    \item 排序顺序: 本文默认升序排序。
    \item 数据类型: 本文默认使用 \texttt{Integer} 类型的数据;其他实现了 \texttt{Comparable} 接口的数据类型同样有效。
\end{itemize}

此外,本章对一些比较,交换等算法进行了封装,例如 \texttt{CollectionUtils.exchange()} 方法用于交换集合中元素位置 \texttt{MathUtils.isBig()} 用于判断元素大小。这样能够有效避免不同数据类型具体操作上的不同。这些方法通过名字就能知道功能,故不作详细解释。