\section{初级排序算法}

初级排序算法指: 冒泡排序,选择排序,插入排序。这三个排序算法的时间复杂度都是 $O(n^2)$,且思路简单不适用于大数据处理。

\subsection{冒泡排序}

动态演示: \url{https://algorithm-visualizer.org/brute-force/bubble-sort}

冒泡排序的核心思想是遍历数组,比较前后两个元素(a[i], a[i+1]),如果前一个元素大于后一个元素则交换这两个元素,这样一轮遍历能保证尾部元素是最大的元素。第二次遍历保证倒数第二个元素是第二大的元素,以此类推,\texttt{n-1} 次遍历后排序完成。

当然也可以从后往前遍历,保证首部元素最小。

\begin{Java}
public static void bubbleSort(Comparable[] a) {
    // 遍历结束位置 i
    for (int i = a.length -1; i > 0; i--) {
        // 从 0 到 i - 1 位置开始遍历
        for (int j = 0; j < i ; j++) {
            if (MathUtils.isBig(a[j], a[j+1]))
                CollectionUtils.exchange(a, j, j+1);
        }
    }
}
\end{Java}

冒泡排序的特点是需要频繁地交换位置,至多 $n^2/2$ 次 但是交换位置的过程中元素大小顺序不会改变。

\begin{itemize}
  \item 时间复杂度: $O(n^2)$ ($O(n^2/2)$)。
  \item 空间复杂度: $O(1)$ 无需格外空间。
  \item 稳定性: 稳定。
\end{itemize}

\subsection{选择排序}

动态演示: \url{https://algorithm-visualizer.org/brute-force/selection-sort}

选择排序是最简单的排序算法之一,他需要对数组进行两次遍历,它的思想是优先找出最小的元素,放在第一个位置,其次找到第二小的元素,放在第二个位置...直到最后一个最大的元素放在最后。

\begin{Java}
public static void selectSort(Comparable[] a) {
    // 第一次 i 遍历 [0, length-1]
    for (int i = 0; i < a.length - 1; i++) {
        // 记录最小值
        int min = i;
        // 第二次 j 遍历 [i, length]
        for (int j = i + 1; j < a.length; j++) {
            if (MathUtils.isBig(a[i], a[j]))  // 判断 a[i] > a[j]
              min = j;
        CollectionUtils.exchange(a, i, j);  // 交换位置
    }
}
\end{Java}

选择排序最符合常规思维,且至多交换 $n$ 次位置。

\begin{itemize}
  \item 时间复杂度: $O(n^2)$ ($O(n^2/2)$)。
  \item 空间复杂度: $O(1)$ 无需格外空间。
  \item 稳定性: 不稳定,$[5_1,8,5_2,2] \rightarrow [2,8,5_2,5_1]$
\end{itemize}

\subsection{插入排序}

动态演示: \url{https://algorithm-visualizer.org/brute-force/insertion-sort}

插入排序类似于纸牌的插牌过程,选择第 i 个元素与前 i-1 个元素比较,直至找到合适的位置插入。

\begin{Java}
public static void insertSort(Comparable[] a) {
    for (int i = 1; i < a.length; i++) {
        Comparable temp = a[i];
        int index = i-1;
        while (index >= 0 && MathUtils.isBig(a[index], temp)) {
            a[index+1] = a[index];
            index--;
        }
        a[index+1] = temp;
    }
}
\end{Java}

插入排序的特点是前 i 个元素有序。

\begin{itemize}
  \item 时间复杂度: $O(n^2)$ ($O(n^2/2)$)。
  \item 空间复杂度: $O(1)$ 无需格外空间。
  \item 稳定性: 稳定。
\end{itemize}

冒泡排序与选择排序无法避免比较,只能减少交换。但是插入排序不同,根据数据模型的不同,新元素的插入位置有所不同,对于前 i 个元素,新元素插入位置在 0-i 之间,如果模型良好,只需要极少数的比较和交换就能完成插入过程。下面看看希尔排序是怎么优化的。

\subsection{希尔排序}

动态演示: \url{https://algorithm-visualizer.org/brute-force/shellsort}

希尔排序是对插入排序的优化,也是初级算法中唯一一个时间复杂度不为 $O(n^2)$ 的算法。

我们知道,插入排序的元素插入过程和数据模型的好坏有很大关系,如果新的元素正好或接近为前 i 个中最大的元素,就可以减少甚至无需插入。希尔排序引入步长来解决这个问题。

希尔排序的过程如下:
\begin{itemize}
  \item 获取步长 h,一般不超过 n/3。例如 50 个元素,取16。
  \item 对每 h 个元素进行插入排序, 例如 [0,16,32],[1,17,33]... 
  \item 改变步长 h = h/3,重复上述过程。
  \item 直到(也必须到) h=1, 完成排序。
\end{itemize}

希尔排序必须保证最后 h 为 1,否则无法保证所有元素有序。

希尔排序在两个方面优化了插入排序:
\begin{itemize}
  \item 初始步长较大,只有少量 ($\lceil n/h \rceil$) 元素需要排序。由于元素少,每个元素的插入速度非常快,至多比较 ($\lceil n/h \rceil$) 次。
  \item 数据模型优化,由于对步长内元素已经进行了排序,在后续插入排序过程中,每个所插入的元素需要移动的距离少。
\end{itemize}

\begin{Java}
public static void shellSort(Comparable[] a) {
    int h = 1;
    while (h < a.length / 3) h = h * 3 + 1;
    while (h >= 1) {
        stepSort(a, h);
        h = h / 3;
    }
}

private static void stepSort(Comparable[] a, int h) {
    for (int i = h; i < a.length; i++) {
        int j = i;
        while (j >= h && MathUtils.isBig(a[j - h], a[j])) {
            CollectionUtils.exchange(a, j, j - h);
            j -= h;
        }
    }
}
\end{Java}

希尔排序的初始步长 h 有很多种取法,究竟哪一种是最优解尚无定论,一般采用上述方式获取并每次将步长缩小三倍(所需排序的元素扩大三倍)。

\begin{itemize}
  \item 时间复杂度: $O(n^{1.3})$ (介于 $O(n^1)$ 与 $O(n^2)$ 之间)。
  \item 空间复杂度: $O(1)$ 无需格外空间。
  \item 稳定性: 不稳定。
\end{itemize}

\newpage