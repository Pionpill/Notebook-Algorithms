\section{快速排序}

快速排序是使用最广泛的排序方法,同时也是针对绝大部分数据类型都适用的速度较快的一种排序方法。

\subsection{分治思想}

快速排序和归并排序都采用了分治的思想。不同的是,归并排序将两个子数组排序后再合并,而快速排序是通过元素\textbf{划分}子数组(一边元素大,一边元素小),两个子数组都有序整个数组也自然有序了。

归并排序递归调用发生在处理整个数组之前,快速排序则发生在处理整个数组之后。

简而言之,快速排序选择某一个元素(通常是最左边的元素)作为划分点,比其小的元素放在左边,比起大的元素放在右边。这样划分之后,再对左右两个子数组进行递归划分。

\begin{figure}[H]
  \small
  \centering
  \begin{tikzpicture}[font=\small]
      \begin{scope}
          \node[align=right] (text) at (0,0) {原始数组(1)};
          \node (1) at (2,0) {Q};
          \node (2) at (3,0) {U};
          \node (3) at (4,0) {I};
          \node (4) at (5,0) {C};
          \node (5) at (6,0) {K};
          \node (6) at (7,0) {S};
          \node (7) at (8,0) {O};
          \node (8) at (9,0) {R};
          \node (9) at (10,0) {T};
      \end{scope}
      \draw [-Stealth](6,-0.25) -- (6,-1.25) node[midway, left] {选择元素 Q 进行划分};
      \begin{scope}[yshift=-1.5cm]
          \node[align=right] (text) at (0,0) {划分数组(2)};
          \begin{scope}[color=red]
            \node (1) at (2,0) {I};
            \node (1) at (3,0) {C};
            \node (1) at (4,0) {K};
            \node (1) at (5,0) {O};
          \end{scope}
          \node (5) at (6,0) {Q};
          \begin{scope}[color=blue]
            \node (6) at (7,0) {U};
            \node (7) at (8,0) {S};
            \node (8) at (9,0) {R};
            \node (9) at (10,0) {T};
          \end{scope}
      \end{scope}
      \draw [-Stealth](6,-1.75) -- (6,-2.75) node[midway, left] {递归划分};
      \begin{scope}[yshift=-3cm]
        \node[align=right] (text) at (0,0) {左右递归(3)};
        \begin{scope}[color=red]
          \node (1) at (2,0) {C};
          \node (1) at (3,0) {I};
          \node (1) at (4,0) {K};
          \node (1) at (5,0) {O};
        \end{scope}
        \node (5) at (6,0) {Q};
        \begin{scope}[color=gray!60!white]
          \node (6) at (7,0) {U};
          \node (7) at (8,0) {S};
          \node (8) at (9,0) {R};
          \node (9) at (10,0) {T};
        \end{scope}
      \end{scope}
      \begin{scope}[yshift=-3.5cm]
        \begin{scope}[color=gray!60!white]
          \node (1) at (2,0) {C};
          \node (1) at (3,0) {I};
          \node (1) at (4,0) {K};
          \node (1) at (5,0) {O};
        \end{scope}
        \node (5) at (6,0) {Q};
        \begin{scope}[color=blue]
          \node (6) at (7,0) {R};
          \node (7) at (8,0) {S};
          \node (8) at (9,0) {T};
          \node (9) at (10,0) {U};
        \end{scope}
      \end{scope}
      \draw [-Stealth](6,-3.75) -- (6,-4.75);
      \begin{scope}[yshift=-5cm]
        \node[align=right] (text) at (0,0) {结果(4)};
        \node (1) at (2,0) {C};
        \node (2) at (3,0) {I};
        \node (3) at (4,0) {K};
        \node (4) at (5,0) {O};
        \node (5) at (6,0) {Q};
        \node (6) at (7,0) {R};
        \node (7) at (8,0) {S};
        \node (8) at (9,0) {T};
        \node (9) at (10,0) {U};
      \end{scope}
  \end{tikzpicture}
  \caption{快速排序示意图}
  \label{fig:快速排序示意图}
\end{figure}

递归我们在上一节已经说明过了,现在只需要得到 $(1) \rightarrow (2)$ 的算法即可,递归过程如下:

\begin{Java}
public static void quickSort(Comparable[] arr) {
    sort(arr, 0, arr.length - 1);
}

private static void sort(Comparable[] arr, int low, int high) {
    if (low >= high)
        return;
    int mid = partition(arr, low, high);
    sort(arr, low, mid - 1);
    sort(arr, mid + 1, high);
}
\end{Java}

\subsection{核心算法}

\subsubsection{双指针法}

最简单的方法是采用双指针:
\begin{itemize}
  \item 第一个指针从左到右移动,直到遇到比划分点元素大的元素,停止。
  \item 第二个指针从右到左移动,直到遇到比划分点元素小的元素,停止。
  \item 两个指针都停止后,交换两个指针的元素。
  \item 重复上述过程,直到指针碰撞。
\end{itemize}

\begin{figure}[H]
    \small
    \centering
    \begin{tikzpicture}[font=\small]
      \begin{scope}[color=white, ultra thick, minimum height=.5cm,minimum width=.5cm]
        \begin{scope}
          \node[fill=black] (1) at (0,0) {Q};
          \node[color=black] (p1) at (1,1) {p1};
          \node[fill=red] (2) at (1,0) {U};
          \draw[-Stealth, color=black, thin] (p1) -- (2);
          \node[fill=red] (3) at (2,0) {I};
          \node[fill=red] (4) at (3,0) {C};
          \node[fill=red] (5) at (4,0) {K};
          \node[fill=red] (6) at (5,0) {S};
          \node[fill=red] (7) at (6,0) {O};
          \node[fill=red] (8) at (7,0) {R};
          \node[color=black] (p2) at (8,1) {p2};
          \node[fill=red] (9) at (8,0) {T};
          \draw[-Stealth, color=black, thin] (p2) -- (9);
        \end{scope}
        \draw[-Stealth, color=black, thin] (0,-0.5) -- (0,-1.5) node [midway, left] {第一轮移动};
        \begin{scope}[yshift=-2cm]
          \node[fill=black] (1) at (0,0) {Q};
          \node[color=black] (p1) at (1,1) {p1};
          \node[fill=green!60!black] (2) at (1,0) {U};
          \draw[-Stealth, color=black, thin] (p1) -- (2);
          \node[fill=red] (3) at (2,0) {I};
          \node[fill=red] (4) at (3,0) {C};
          \node[fill=red] (5) at (4,0) {K};
          \node[fill=red] (6) at (5,0) {S};
          \node[color=black] (p2) at (6,1) {p2};
          \node[fill=green!60!black] (7) at (6,0) {O};
          \node[fill=blue!60] (8) at (7,0) {R};
          \node[fill=blue!60] (9) at (8,0) {T};
          \draw[-Stealth, color=black, thin] (p2) -- (7);
        \end{scope}
        \draw[-Stealth, color=black, thin] (0,-2.5) -- (0,-3.5) node [midway, left] {交换元素位置};
        \begin{scope}[yshift=-4cm]
          \node[fill=black] (1) at (0,0) {Q};
          \node[color=black] (p1) at (1,1) {p1};
          \node[fill=blue!60] (2) at (1,0) {O};
          \draw[-Stealth, color=black, thin] (p1) -- (2);
          \node[fill=red] (3) at (2,0) {I};
          \node[fill=red] (4) at (3,0) {C};
          \node[fill=red] (5) at (4,0) {K};
          \node[fill=red] (6) at (5,0) {S};
          \node[color=black] (p2) at (6,1) {p2};
          \node[fill=blue!60] (7) at (6,0) {U};
          \node[fill=blue!60] (8) at (7,0) {R};
          \node[fill=blue!60] (9) at (8,0) {T};
          \draw[-Stealth, color=black, thin] (p2) -- (7);
        \end{scope}
        \draw[-Stealth, color=black, thin] (0,-4.5) -- (0,-5.5) node [midway, left] {经过多轮移动};
        \begin{scope}[yshift=-6cm]
          \node[fill=black] (1) at (0,0) {Q};
          \node[fill=blue!60] (2) at (1,0) {O};
          \node[fill=blue!60] (3) at (2,0) {I};
          \node[fill=blue!60] (4) at (3,0) {C};
          \node[fill=green!60!black] (5) at (4,0) {K};
          \node[fill=green!60!black] (6) at (5,0) {S};
          \node[fill=blue!60] (7) at (6,0) {U};
          \node[fill=blue!60] (8) at (7,0) {R};
          \node[fill=blue!60] (9) at (8,0) {T};
          \node[color=black] (p1) at (5,1) {p1};
          \draw[-Stealth, color=black, thin] (p1) -- (6);
          \node[color=black] (p2) at (4,1) {p2};
          \draw[-Stealth, color=black, thin] (p2) -- (5);
        \end{scope}
        \draw[-Stealth, color=black, thin] (0,-6.5) -- (0,-7.5) node [midway, left] {p1>p2, 交换};
        \begin{scope}[yshift=-8cm]
          \node[fill=blue!60] (1) at (0,0) {K};
          \node[fill=blue!60] (2) at (1,0) {O};
          \node[fill=blue!60] (3) at (2,0) {I};
          \node[fill=blue!60] (4) at (3,0) {C};
          \node[fill=black] (5) at (4,0) {Q};
          \node[fill=blue!60] (6) at (5,0) {S};
          \node[fill=blue!60] (7) at (6,0) {U};
          \node[fill=blue!60] (8) at (7,0) {R};
          \node[fill=blue!60] (9) at (8,0) {T};
          \node[color=black] (p1) at (5,1) {p1};
          \draw[-Stealth, color=black, thin] (p1) -- (6);
          \node[color=black] (p2) at (4,1) {p2};
          \draw[-Stealth, color=black, thin] (p2) -- (5);
        \end{scope}
        \node[color = black] at (4,-9) {返回 p2};
      \end{scope}
    \end{tikzpicture}
    \caption{双指针快速排序}
    \label{fig:双指针快速排序}
\end{figure}

此外我们必须考虑两种极端情况,一是数组元素有序,例如 1,2,3,4; 4,3,2,1。而是数组元素最小,例如 1,2; 2,1。
\begin{itemize}
  \item 在第一种情况下,如果是 1,2,3,4: p2 最终将指向 1, 然后 1 与本身交换,这没什么问题; 如果是 4,3,2,1: p1 会直接超出范围,这是不允许的,因此 p1 <= high。
  \item 在第二种情况下, 如果是  1,2: p2 指向 1 与自己交换没有问题;如果是 2,1: p1 指向1, p2 指向1, 1 与 2 交换,也没什么问题。
\end{itemize}

因此我们只需要判断 p1 <= high 即可,p2 >= low 没有必要。

\begin{Java}
private static int partition(Comparable[] arr, int low, int high) {
    Comparable v= arr[low];
    int p1 = low + 1;
    int p2 = high;
    while (true) {
        while (MathUtils.isLess(arr[p1], v)) {
            if (p1 == high)
                break;
            p1++;
        }
        while (MathUtils.isBig(arr[p2], v)) {
            p2--;
        }
        if (p1 >= p2) break;
        CollectionUtils.exchange(arr, p1, p2);
    }
    CollectionUtils.exchange(arr, low, p2);
    return p2;
}
\end{Java}

和归并类似,至多 $\lceil \log_2 n \rceil$ 此运算即可完成递归。

\subsubsection{三向切分}

快速排序发展的过程中遇到一个棘手的问题,如果有太多值相同的元素,快速排序会傻乎乎地将它们随机放在一边,如果能将这些值集中在中间位置,将显著提高快速排序针对这类数据模型的计算速度。